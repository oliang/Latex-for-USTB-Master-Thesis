\chapter{绪论}
\pagenumbering{arabic}
\section{研究背景及意义}
随着电子商务的发展,网上银行、网上合同、电子签名等的应用越来越广泛,网络已经成为我们生活中不可或缺的一部分。电子商务在给我们的工作生活带来便捷的同时,也存在着安全隐患。举个简单的例子,Hash密码算法一直在这些领域中起着身份验证、口令加密、防篡改和重放攻击等作用,目前的用户口令认证机制中,系统将用户的口令进行Hash算法加密后存储,以便下次检验用户身份,如果攻击Hash算法得到了口令,可想而知对整个系统的安全造成了多大的威胁。

对密码进行分析主要是为了发现加密算法、密钥或密码系统的弱点,以完善加密过程,更有利于信息的安全。另一方面,是为了掌握密码分析者或破译者攻击密码的方法,找出其方法的漏洞,便于预防他们的攻击。同时也是为了更进一步提高广大计算机用户的安全意识和知识水平,减少针对系统的非法入侵和攻击带来的损失。我们知道在整个密码系统中,最有价值的信息就是密钥,绝大部分系统的密钥是用Hash函数来保护的,因此,针对密钥的攻击分析是密码分析领域的一个非常有价值的研究的课题。

\section{国内外研究现状与进展}
在1980年,Martin Hellman\cite{hellman}提出了一个“时间空间折中”的密码分析算法,使用了预先计算好并保存在内存和磁盘里面的数据,减少了密码分析需要的时间。这个算法在1982年被Ri
vest提出改进,减少了密码分析过程中所需要的存储空间。  

2003年7月瑞士洛桑联邦技术学院的Philippe Oechslin公布了一些实验数据,他及其所属的安全及密码学实验室(LASEC)采用了时间空间折中算法,使得密码破解的效率大大提高。他们开发的O
phcrack项目可以将一个操作系统的用户登录密码破解速度由1分41秒,提升到13.6秒\cite{PO}。该项目提供了一个破解视窗作业系统下的LAN Manager散列(比如hash文件)的程序,作者免费提
供了一些Rainbow table,可以在短至几秒内破解最多14个英文字母的密码,有99.9%的成功率。从2.3版开始可以破解 NT 散列,这功能对已经关闭 LAN Manager 散列的系统(Windows Vista的预订设定)或是长于14个字母的密码特别有用。

同年project-rainbowcrack项目开始立项,该项目基于Philippe Oechslin提出的彩虹表,用C++基本实现了对MD5、SHA-1算法的低位数低密钥空间的破解\cite{zhu}。接着出现了一个分布式彩虹表项目F
ree Rainbow Tables,这个项目的分布式系统是基于伯克利开放式网络计算平台(BOIN)。

在我国密码分析还处于初级阶段,由于软、硬件及技术等各种原因,大部分密码分析方法还处于理论阶段。目前,已经出现了各种各样的密码分析系统,都是针对某种加密方法进行分析的,功能和方法上还具有一定的局限性。2004年8月,在美国加州圣芭芭拉召开的国际密码大会上,山东大学王小云教授在会议上首次宣布了她及她的研究小组近年来的研究成果——对MD5、HAVAL-128、MD4和RIPEMD等四个著名密码算法的破译结果。2008年国际密码学家Lenstra利用王小云提供的MD5碰撞,伪造了符合X.509标准的数字证书,说明了MD5的破译已经不是理论破译结果,而是可以导致实际的攻击,目前SHA-1在理论上已经被破译,离实际应用也为期不远。目前国内已经有对基于时空折衷算法的Word文档破解研究\cite{word}和对DES密码算法的彩虹攻击技术及其GPU实现\cite{des}两篇与彩虹表算法相关的文献。
\section{本文研究现状}
本文研究的主要内容就是基于时间空间折中算法的Hash密钥分析。主要采用彩虹表进行Hash算法破解,并进一步对时空折中算法的研究和优化,开发出基于CUDA\cite{nvidia}模型的彩虹表算法实现。主要研究成果有:
\begin{enumerate}
\item 优化彩虹表算法参数,减少破解时间;
\item 优化彩虹表的数据结构,减少表的存储空间;
\item 利用GPU高性能并行运算提高破解速度。
\end{enumerate}
\section{论文组织结构}
本文共分六章,全文结构安排如下:

第一章 \quad 绪论。介绍了本课题的研究背景及意义、国内外研究现状与进展、研究现状以及本文组织结构。

第二章 \quad 相关知识背景。

第三章\quad  时空折中算法。

第四章 \quad 彩虹表算法的实现。

第五章 \quad 彩虹表算法的优化。

第六章 \quad 本文总结与展望。



